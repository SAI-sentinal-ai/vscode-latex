\documentclass{article}
\usepackage{imfellEnglish}
\usepackage[T1]{fontenc}

\title{St. Augustine of Hippo\thanks{This is a footnote}}
\author{\textit{Eugène Portalié}}
\date{\small{1907}}

\begin{document}

\maketitle

\begin{abstract}
The great St. Augustine's life is unfolded to us in documents of unrivaled richness, and of no great character of ancient times have we information comparable to that contained in the \textit{Confessions}, which relate the touching story of his soul, the \textit{Retractations}, which give the history of his mind, and the \textit{Life of Augustine}, written by his friend Possidius, telling of the saint's apostolate.

We will confine ourselves to sketching the three periods of this great life: (1) the young wanderer's gradual return to the Faith; (2) the doctrinal development of the Christian philosopher to the time of his episcopate; and (3) the full development of his activities upon the Episcopal throne of Hippo.
\end{abstract}

% this is a comment
\section{From his birth}
Augustine was born at Tagaste on 13 November, 354. Tagaste, now Souk-Ahras, about 60 miles from Bona (ancient Hippo-Regius), was at that time a small free city of proconsular Numidia which had recently been converted from Donatism. Although eminently respectable, his family was not rich, and his father, Patricius, one of the curiales of the city, was still a pagan. However, the admirable virtues that made Monica the ideal of Christian mothers at length brought her husband the grace of baptism and of a holy death, about the year 371.

Augustine received a Christian education. His mother had him signed with the cross and enrolled among the catechumens. Once, when very ill, he asked for baptism, but, all danger being soon passed, he deferred receiving the sacrament, thus yielding to a deplorable custom of the times. His association with ``men of prayer'' left three great ideas deeply engraven upon his soul: a Divine Providence, the future life with terrible sanctions, and, above all, Christ the Saviour. ``From my tenderest infancy, I had in a manner sucked with my mother's milk that name of my Saviour, Thy Son; I kept it in the recesses of my heart; and all that presented itself to me without that Divine Name, though it might be elegant, well written, and even replete with truth, did not altogether carry me away'' (Confessions I.4).

\paragraph{Outline}
First we start with a little example of the article class, which is an 
important documentclass. But there would be other documentclasses like 
book \ref{book}, report \ref{report} and letter \ref{letter} which are 
described in Section \ref{documentclasses}. Finally, Section 
\ref{conclusions} gives the conclusions.



\section{Documentclasses} \label{documentclasses}

\begin{itemize}
\item article
\item book 
\item report 
\item letter 
\end{itemize}


\begin{enumerate}
\item article
\item book 
\item report 
\item letter 
\end{enumerate}

\begin{description}
\item[article\label{article}]{Article is \ldots}
\item[book\label{book}]{The book class \ldots}
\item[report\label{report}]{Report gives you \ldots}
\item[letter\label{letter}]{If you want to write a letter.}
\end{description}


\section{Conclusions}\label{conclusions}
There is no longer \LaTeX{} example which was written by \cite{doe}.


\begin{thebibliography}{9}
\bibitem[Doe]{doe} \emph{First and last \LaTeX{} example.},
John Doe 50 B.C. 
\end{thebibliography}

\end{document}